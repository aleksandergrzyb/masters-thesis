% **************************************************
% Clean Thesis
% -- A LaTeX Style for Thesis Documents --
%
% Copyright (C) 2011-2014 Ricardo Langner
% **************************************************
%
% Readme:
% ----------------------------------------
% *** Clean, Simple, Elegant ***
% "Clean Thesis" is a LaTeX style for thesis documents, developed
% for my diplom thesis (Diplomarbeit). The style can be understood
% as my personal compromise - a typical clean looking scientific
% document combined and polished with minor beautifications.
%
% The design of this "Clean Thesis" style is inspired
% by user guide documents from Apple Inc.
%
% Note: If you are looking for an exact and correct style regarding
% typographic rules, please have a look at the "Classic Thesis Style"
% (see http://www.miede.de/index.php?page=classicthesis).
%
% *** Donation = Postcard ***
% Based on the idea of Andr\'e Miede: If you like the "Clean Thesis"
% style I would be very pleased about a donation in the form of a
% POSTCARD. You can find my address in the file Clean-Thesis.pdf.
% I am going to collect all postcards and exhibit them at the website
% I mentioned.
%
% *** Idea and Inspiration ***
% The idea of providing my customized style for thesis documents
% passed through my mind while writing my own thesis. Motivated and
% inspired by the superb "Classic Thesis Style"
% (see http://www.miede.de/index.php?page=classicthesis) by Andr\'e Miede
% (thanks to Andr\'e for doing a great job) I decided to collect all
% design and style related functionality in a separate LaTeX style and
% provide this style to other thesis writers.
%
%
% License Information:
% ----------------------------------------
% "Clean Thesis" is free software: you can redistribute it and/or modify
% it under the terms of the GNU General Public License as published by
% the Free Software Foundation, either version 3 of the License, or
% (at your option) any later version.
%
% "Clean Thesis" is distributed in the hope that it will be useful,
% but WITHOUT ANY WARRANTY; without even the implied warranty of
% MERCHANTABILITY or FITNESS FOR A PARTICULAR PURPOSE.  See the
% GNU General Public License for more details.
%
% You should have received a copy of the GNU General Public License
% along with this program.  If not, see <http://www.gnu.org/licenses/>.
% **************************************************


% **************************************************
% Document Class Definition
% **************************************************
\documentclass[%
	paper=A4,					% paper size --> A4 is default in Germany
	twoside=true,				% onesite or twoside printing
	openright,					% doublepage cleaning ends up right side
	parskip=full,				% spacing value / method for paragraphs
	chapterprefix=true,			% prefix for chapter marks
	11pt,						% font size
	headings=normal,			% size of headings
	bibliography=totoc,			% include bib in toc
	listof=totoc,				% include listof entries in toc
	titlepage=on,				% own page for each title page
	captions=tableabove,		% display table captions above the float env
	draft=false,				% value for draft version
]{scrreprt}%

\usepackage{polski}
\usepackage[utf8]{inputenc} % defines file's character encoding
\usepackage[T1]{fontenc} % babel system, adjust the language of the content
\usepackage[polish]{babel}
\usepackage{float}
\usepackage{amsmath}
\usepackage{url}

\pdfinclusioncopyfonts=1
% **************************************************
% Debug LaTeX Information
% **************************************************
%\listfiles

% **************************************************
% Information and Commands for Reuse
% **************************************************
\newcommand{\thesisTitle}{Algorytmy korejestracji przestrzennej dla angiograficznego obrazowania w optycznej tomografii koherencyjnej}
\newcommand{\thesisName}{Aleksander Grzyb}
\newcommand{\thesisDate}{2015}
\newcommand{\thesisProjectName}{Praca zrealizowana w ramach projektu ,,Rozwój interferometrycznych metod optycznych do badania dynamiki układów biologicznych'' (RIMO-BIOL), NCBiR PBS1/A9/20/2013.}

\newcommand{\thesisFirstSupervisor}{dr hab.~inż.~Krzysztof Krawiec}

\newcommand{\thesisUniversity}{\protect{Politechnika Poznańska}}
\newcommand{\thesisUniversityDepartment}{Wydział Informatyki}
\newcommand{\thesisUniversityCity}{Poznań}

% **************************************************
% Load and Configure Packages
% **************************************************

\usepackage[					% clean thesis style
	figuresep=colon,%
	sansserif=false,%
	hangfigurecaption=false,%
	hangsection=true,%
	hangsubsection=true,%
	colorize=bw,%
	colortheme=bluemagenta,%
]{cleanthesis}

\definecolor{links}{HTML}{006598}

\hypersetup{					% setup the hyperref-package options
	pdftitle={\thesisTitle},	% 	- title (PDF meta)
	pdfauthor={\thesisName},	% 	- author (PDF meta)
	plainpages=false,			% 	-
	colorlinks=false,			% 	- colorize links?
	linkcolor=links,
	citecolor=links,
	urlcolor=links,
	pdfborder={0 0 0},			% 	-
	breaklinks=true,			% 	- allow line break inside links
	bookmarksnumbered=true,		%
	bookmarksopen=true			%
}

% \raggedbottom

% **************************************************
% Document CONTENT
% **************************************************
\begin{document}

% --------------------------
% rename document parts
% --------------------------
%\renewcaptionname{ngerman}{\figurename}{Abb.}
%\renewcaptionname{ngerman}{\tablename}{Tab.}
\renewcaptionname{polish}{\figurename}{Rys.}
\renewcaptionname{polish}{\tablename}{Tab.}
% \renewcaptionname{english}{\figurename}{Fig.}
% \renewcaptionname{english}{\tablename}{Tab.}

% --------------------------
% Front matter
% --------------------------
\pagestyle{empty}				% no header or footers
% !TEX root = ../thesis-example.tex
%
% ------------------------------------  --> cover title page
\begin{titlepage}
	\pdfbookmark[0]{Cover}{Cover}
	\flushright
	\hfill
	\vfill
	{\LARGE\thesisTitle} \par
	\rule[5pt]{\textwidth}{.4pt} \par
	{\Large\thesisName}
	\vfill
	\textit{\large\thesisDate} \\
\end{titlepage}


% ------------------------------------  --> main title page
\begin{titlepage}
	\pdfbookmark[0]{Titlepage}{Titlepage}
	\tgherosfont
	\centering

	{\Large \thesisUniversity} \\[4mm]
	\includegraphics[width=12cm]{gfx/logopp} \\[2mm]
	\textsf{\thesisUniversityDepartment} \\

	\vfill
	{\LARGE \color{ctcolormain}\textbf{\thesisTitle} \\[10mm]}
	{\Large \thesisName} \\

	\vfill
	\hspace*{15pt}

	\begin{minipage}[t]{.27\textwidth}
		\raggedleft
		\textit{Promotor}
	\end{minipage}
	\hspace*{10pt}
	\begin{minipage}[t]{.65\textwidth}
		\thesisFirstSupervisor
	\end{minipage} \\[10mm]

	\thesisDate \\

\end{titlepage}


% ------------------------------------  --> lower title back for single page layout
\hfill
\vfill
{
	\small
	\textbf{\thesisName} \\
	\textit{\thesisTitle} \\
	\thesisDate \\
	Promotor: \thesisFirstSupervisor \\
	\textbf{\thesisUniversity} \\
	\thesisUniversityDepartment \\
	\thesisUniversityCity \\
}
		% INCLUDE: all titlepages
\cleardoublepage

\pagestyle{plain}				% display just page numbers
% !TEX root = ../thesis.tex
%

\chapter*{Podziękowania}
\label{sec:acknowledgement}
\vspace*{-10mm}

Składam serdeczne podziękowania dla mojego promotora, profesora Krzysztofa Krawca za wsparcie, inspirację oraz poświęcony czas w trakcie realizacji niniejszej pracy. Również chciałbym podziękować Danielowi Rumińskiemu za owocną współpracę oraz wsparcie merytoryczne.
 % INCLUDE: acknowledgement
\cleardoublepage

\setcounter{tocdepth}{2}		% define depth of toc
\tableofcontents				% display table of contents\tabularnewline
\cleardoublepage

% --------------------------
% Body matter
% --------------------------
\pagenumbering{arabic}			% arabic page numbering
\setcounter{page}{1}			% set page counter
\pagestyle{maincontentstyle} 	% fancy header and footer

% !TEX root = ../thesis-example.tex
%
\chapter{Wstęp}
\label{sec:wstep}

\section{Opis problemu}
\label{sec:wstep:opis_problemu}

\section{Cel i zakres pracy}
\label{sec:wstep:cel_i_zakres}

\section{Projekt RIMO}
\label{sec:wstep:rimo}
% !TEX root = ../thesis-example.tex
%
\chapter{Obrazowanie OCT}
\label{sec:obrazowanie_oct}

% Wstęp do OCT

\textbf{Optyczna tomografia koherencyjna} (ang. \textit{optical coherence tomography, OCT}) jest metodą umożliwiającą nieinwazyjne oraz \textit{in vivo} przechwycenie obrazu wnętrza tkanki biologicznej. Zasada działania OCT opiera się na wykorzystaniu fal świetlnych. Dzięki temu rozdzielczość obrazów jest o wiele wyższa niż w ultrasonografii (wykorzystanie fal dźwiękowych), czy rezonansie magnetycznym (wykorzystanie pola magnetycznego). Następnym powodem dużej popularności OCT w medycynie jest bezkontaktowe badanie pacjenta oraz brak wymogu przygotowania pacjenta przed badaniem. 

W projekcie RIMO OCT zostało wykorzystane do uzyskania szczegółowych obrazów naczyń krwionośnych siatkówki oka. Rysunek \ref{fig:obrazowanie_oct:bscan_vessels} przedstawia przykładowe obrazy siatkówki oka uzyskane dzięki wykorzystaniu OCT. Jednym z tych przykładowych obrazów jest angiografia siatkówki oka, która jest obrazem wejściowym do algorytmów omawianych w niniejszej pracy. Sposób powstania angiografii z danych OCT jest wyjaśniony w podrozdziale \ref{sec:obrazowanie_oct:angiografia_oct}, natomiast ogólna zasada działania OCT jest wyjaśniona w podrozdziale \ref{sec:obrazowanie_oct:metoda_oct}. Na koniec rozdziału w podrozdziale \ref{sec:obrazowanie_oct:zastosowania_oct} zostaną opisane zastosowania OCT.

\begin{figure}[htb]
	\centering
	\includegraphics[width=\textwidth]{gfx/bscan_vessels}
	\caption{\textbf{Lewy obraz:} Dwuwymiarowy przekrój siatkówki oka (B-skan). Obraz został uzyskany poprzez połączenie jednowymiarowych A-skanów, które zawierają informację o strukturze tkanki w głąb siatkówki oka. \textbf{Prawy obraz:} Angiografia siatkówki oka uzyskana dzięki przetworzeniu danych z OCT.}
	\label{fig:obrazowanie_oct:bscan_vessels}
\end{figure}

% Wyjaśnia zasadę działania OCT, bez wchodzenia w duże teoretyczne detale

\section{Zasada działania OCT}
\label{sec:obrazowanie_oct:metoda_oct}

Jednym z najważniejszych parametrów metod tomografii w medycynie jest ich rozdzielczość. Optyczna tomografia koherencyjna przechwytuje obrazy wnętrza tkanki poprzez wykorzystanie fal świetlnych. OCT za pomocą generatora wytwarza falę świetlną, która jest skierowana na tkankę pacjenta. Następnie po odbiciu fali poprzez tkankę wiązka jest przechwycona przez detektor. Jedną z dostępnych metod, która umożliwiłaby zlokalizowanie miejsca odbicia fali byłoby zmierzenie czasu pomiędzy wygenerowaniem fali, a zarejestrowaniem jej przez detektor (mechanizm stosowany np. w ultrasonografii z wykorzystaniem fal dźwiękowych), natomiast prędkość światła (\(3\times10^8 \frac{m}{s}\)) wyklucza zastosowanie tego mechanizmu. Zjawisko, które umożliwia dokładne zlokalizowanie miejsca odbicia to \textbf{interferencja światła o niskiej spójności}.

\begin{figure}[htb]
	\centering
	\includegraphics[width=\textwidth]{gfx/oct_phases}
	\caption{Kolejne etapy działania metody OCT. \textbf{(1)} - Etap początkowy. \textbf{(2)} - Źródło wyemitowało wiązkę światła. \textbf{(3)} - Fala rozdzieliła się za pomocą interferometru na wiązkę referencyjną (skierowaną na lustro referencyjne) oraz na wiązkę próbki (skierowaną na tkankę). \textbf{(4)} - Wiązki po odbiciu od lustra referencyjnego i tkanki ponownie łączą się za pomocą interferometru. W tej części występuje zjawisko interferencji, które jest zarejestrowane przez detektor.}
	\label{fig:obrazowanie_oct:oct_phases}
\end{figure}

Rysunek \ref{fig:obrazowanie_oct:oct_phases} składa się z bardzo uproszczonych schematów OCT, które obrazują kolejne etapy działania metody. Schematy na rysunku \ref{fig:obrazowanie_oct:oct_phases} składają się z pięciu elementów:

\begin{itemize}

\item \textbf{Źródła} - Źródło (np. dioda superluminescencyjna) światła podczerwonego, które jest falą o niskiej spójności.
\item \textbf{Rozdzielacza wiązek} (na rysunku \ref{fig:obrazowanie_oct:oct_phases} przedstawiony za pomocą przerywanych linii na środku każdego schematu) - Interferometr (np. Michelsona) umożliwiający rozdzielenie fali na dwie wiązki oraz następne ich połączenie.
\item \textbf{Lustro referencyjne} - Lustro, które odbija wiązkę referencyjną. Posiada możliwość oddalania oraz przybliżania się względem interferometru.
\item \textbf{Tkanka} - Badana tkanka, która odbija wiązkę próbki.
\item \textbf{Detektor} - Rejestruje zjawisko interferencji związek.

\end{itemize}

Najbardziej istotnym etapem wymaganym do zrozumienia mechanizmu OCT jest etap (4) pokazany na rysunku \ref{fig:obrazowanie_oct:oct_phases}. W tym kroku wiązka referencyjna i wiązka próbki łączą się i zachodzi zjawisko interferencji. Dzięki temu, że wiązki są falami o niskiej spójności interferencja zachodzi tylko na małej długości (ang. \textit{coherence length}). Odczytując za pomocą detektora charakterystyczny wzorzec interferencji występującej na \textit{coherence length} jesteśmy w stanie wydobyć informacje na temat próbki wnętrza tkanki oraz wiemy dzięki położeniu lustra referencyjnego położenie próbki. Poszczególne badanie głębszych warstw tkanki przedstawia rysunek \ref{fig:obrazowanie_oct:tissue_layers}.

\begin{figure}[htb]
	\centering
	\includegraphics[width=\textwidth]{gfx/tissue_layers}
	\caption{Kolejne etapy badania głębszych warstw tkanki dzięki przesuwaniu lustra referencyjnego. Aktualna pozycja lustra referencyjnego jest zaznaczona kolorem czarnym, natomiast aktualnie badana warstwa tkanki jest zaznaczona kolorem czerwonym.}
	\label{fig:obrazowanie_oct:tissue_layers}
\end{figure}

\subsection{Metoda uzyskania trójwymiarowego obrazu tkanki}

Poprzez poruszenie lustra referencyjnego pomiary przeprowadzane są w głąb tkanki (wzdłuż osi Z). Zbiór pomiarów w głąb tkanki nazywa się A-skanem (obraz jednowymiarowy). Powtarzając ten proces w osi X lub Y i następnie poprzez połączenie sąsiadujących A-skanów otrzymuje się przekrój tkanki zwany B-skanem (obraz dwuwymiarowy). Cały proces uzyskania B-skanów można powtórzyć dla sąsiadujących przekrojów. Poprzez połączenie otrzymanych B-skanów otrzymuje się obraz trójwymiarowy tkanki. Na rysunku \ref{fig:obrazowanie_oct:scan} \cite{Kraus:12} przedstawione są poszczególne skany.

\begin{figure}[htb]
	\centering
	\includegraphics[width=\textwidth]{gfx/scans}
	\caption{\cite{Kraus:12} \textbf{Lewy obraz:} Pojedynczy A-skan wykonany w głąb tkanki. \textbf{Środkowy obraz:} Otrzymany B-skan poprzez połączenie A-skanów. \textbf{Prawy obraz:} Trójwymiarowy obraz tkanki stworzony poprzez połączenie B-skanów.}
	\label{fig:obrazowanie_oct:scan}
\end{figure}

\subsection{OCT w domenie częstotliwości}

Metoda OCT, która wykonuje ruch lustrem referencyjnym jest zwana metodą OCT w domenie czasu (ang. \textit{time-domain OCT, TdOCT}). Alternatywną oraz nowszą metodą od TdOCT jest OCT w domenie częstotliwości (ang. \textit{Fourier-domain OCT, FdOCT}). FdOCT umożliwia 100 razy szybsze \cite{Strong:11} skanowanie w porównaniu do TdOCT. TdOCT jest w stanie wykonać ok. 400 A-skanów w przeciągu sekundy, natomiast FdOCT jest ich w stanie wykonać dziesiątki tysięcy. Szybsze skanowanie poprawia również jakość skanów ze względu na to, że pacjent ma mniejszą szansę poruszenia okiem podczas skanowania (ruch oka w trakcie skanowania przyczynia się do powstania artefaktów ruchu na obrazach OCT). Oprócz poprawy szybkości skanowania FdOCT ma wyższą rozdzielczość w przedziale od 3-7 mikrometrów \cite{Strong:11}. Jest to poprawa względem TdOCT o 8-10 mikrometrów.

Większa szybkość oraz rozdzielczość FdOCT w porównaniu do TdOCT jest możliwa dzięki dwóm modyfikacjom technicznym:

\begin{enumerate}

\item FdOCT jako źródło światła wykorzystuje laser o wysokiej szerokości pasma, co znacząco zwiększa rozdzielczość.
\item FdOCT jako detektor wykorzystuje spektrometr, który przeprowadza analizę widma fali (połączona wiązka referencyjna i próbki), która dotarła do spektrometru z interferometru. Wykonanie transformaty Fouriera na widmie fali tworzy A-skan tkanki. Dzięki tej technice FdOCT wyeliminowało potrzebę ruszania lustrem referencyjnym co znacząco zwiększa szybkość skanowania.

\end{enumerate}

FdOCT posiada również wady. Jest droższy od TdOCT ze względu na wykorzystanie drogiego lasera jako źródła światła co powoduje, że jest wykorzystywany tylko w celach badawczych. Następna wada wynika z szybkości powstawania A-skanów. Szybsze skanowanie może prowadzić do utraty jakości obrazów OCT, natomiast utrata tej jakość może być zmniejszona poprzez użycie techniki przetwarzania sygnałów zwanej \textit{oversampling}.

% Wyjaśnia zasadę powstawania obrazów angiograficznych z danych OCT

\section{Angiografia OCT}
\label{sec:obrazowanie_oct:angiografia_oct}

Angiografia to technika służąca do wizualizacji naczyń krwionośnych i organów ciała. W większości przypadków w medycynie angiografia jest wykonywana poprzez wstrzyknięcie pacjentowi środka kontrastowego, który nie przepuszcza promieni rentgenowskich. Następie wykorzystuje się jedną z technik obrazowania (np. fluoroskopię) opartą o promienie rentgenowskie. Ta metoda ma jedną znaczącą wadę, jako technika inwazyjna może prowadzić do reakcji alergicznych na środek kontrastowy oraz jest przeciwwskazana kobietom w ciąży i dzieciom. Z tego powodu nieustannie poszukiwane są metody, które jednocześnie są nieinwazyjne i tworzą obrazy o jakości porównywalnej do metod inwazyjnych.

Angiografia OCT jest metodą, która opiera się na A-skanach i B-skanach OCT i jest w stanie stworzyć obraz \textit{en face} naczyń krwionośnych siatkówki oka. Przykład takiego obrazu znajduje się na rysunku \ref{sec:obrazowanie_oct:angiografia_oct}.

\subsection{Sposób powstania obrazu angiograficznego}

Obrazy angiograficznie, które zostały użyte w niniejszej pracy powstały za pomocą metody zwanej \textit{Speckle Variance Detection}, która polega na liczeniu wariancji dla każdego piksela pomiędzy sąsiadującymi B-skanami. Ta technika została wykorzystana, ponieważ lokalna wariancja dla obszarów gdzie występuje przypływ krwi (naczynia krwionośne) ma wyższą wartość w porównaniu do obszarów gdzie występuje struktura statyczna. Rezultatem tej metody jest obraz przepływowy, którego piksele mają wartość wariancji sąsiadujących B-skanów. Na rysunku \ref{fig:obrazowanie_oct:speckle_variance} został przedstawiony przykładowy zbiór sąsiadujących B-skanów.

\begin{figure}[H]
	\centering
	\includegraphics[width=4cm]{gfx/speckle_variance}
	\caption{Trzy sąsiadujące B-skany na postawie których powstaje jeden obraz przepływowy.}
	\label{fig:obrazowanie_oct:speckle_variance}
\end{figure}

Wartości intensywności pikseli $V_{jk}$ obrazu przepływowego liczone są na podstawie wartości intensywności pikseli $I_{ijk}$ sąsiadujących $N$ B-skanów za pomocą wzoru (na rysunku \ref{fig:obrazowanie_oct:speckle_variance} $N=3$):

$$ V_{jk} = \frac{1}{N} \displaystyle\sum_{i=1}^{N}(I_{ijk} - I_{mean})^2 $$

Gdzie $j$ i $k$ to indeksy boczne i głębokościowe B-skanu (współrzędne pikseli), a $I_{mean}$ to średnia intensywność zbioru tych samych pikseli, dla których liczona jest wariancja.

Rezultatem połączenia sąsiadujących obrazów przepływowych obliczonych ze wszystkich B-skanów jest trójwymiarowy obraz przepływowy. Następnie poprzez użycie projekcji maksymalnego natężenia (ang. \textit{maximum intensity projection, MIP}), która polega na projekcji woksela o najwyższej intensywności z obrazu 3D na obraz 2D. W przypadku trójwymiarowego obrazu przepływowego będzie to woksel, którego wartość była największą wartością wariancji pikseli sąsiadujących B-skanów. Po zastosowaniu projekcji maksymalnego natężenia na trójwymiarowym obrazie przepływowym otrzymujemy angiograficzny obraz \textit{en face} naczyń krwionośnych.

% Opisuje zastosowania OCT

\section{Zastosowania OCT}
\label{sec:obrazowanie_oct:zastosowania_oct}

Optyczna tomografia koherencyjna ze względu na swoje właściwości (badanie nieinwazyjne oraz \textit{in vivo}) jest metodą, która ma szerokie zastosowanie w medycynie oraz w innych specjalizacjach.

\subsection{OCT w okulistyce}

Najbardziej popularnym zastosowaniem OCT w medycynie jest badanie oka \cite{Fercher03}. Technika OCT umożliwia przechwycenie trójwymiarowych obrazów części oka takich jak dno, czy warstwy przednie. Dzięki temu jest wykorzystywane do diagnozowania takich chorób jak stwardnienie rozsiane, zwyrodnienie plamki żółtej, czy jaskra. 

\subsection{OCT w gastroenterologii i dermatologii}

OCT w porównaniu do innych metod diagnostycznych w medycynie jest techniką nową. Lekarze i naukowcy nieustannie starają się znaleźć nowe zastosowania dla OCT, która jest metodą obiecującą i szybko rozwijającą się. OCT jest potencjalnym kandydatem by w niektórych diagnozach zastąpić konwencjonalną biopsję, która wymaga usunięcia kawałka tkanki z organizmu. Przykładem takiego zastosowania jest badanie struktury błon śluzowych i podśluzowych w układzie pokarmowym. W tym przypadku OCT dostarczyło czyste obrazy, które dostarczają lekarzom dużo diagnostycznej informacji \cite{Rollins:99}. Innym przykładem są próby wykorzystania OCT do wczesnej diagnozy raka skóry, który obecnie również diagnozowany jest poprzez biopsję. W tym przypadku OCT w obecnym stopniu zaawansowania nie jest w stanie dostarczyć na tyle dokładnych danych by stać się jedyną metodą diagnozy. Następnym przykładem zastosowania OCT w dermatologii jest diagnoza zapalnych chorób skóry \cite{Welzel01}.

\subsection{OCT w przemyśle}

OCT wykorzystywane jest również do zastosowań przemysłowych. Umożliwia badanie np. grubości materiałów \cite{walecki2006determining}, czy badanie grubości warstwy pancerza tabletek podczas ich produkcji w przemyśle farmaceutycznym \cite{markl2014device}.

















% !TEX root = ../thesis-example.tex
%
\chapter{Algorytmy korejestracji przestrzennej obrazów OCT}
\label{sec:algorytmy_korejestracji}

\textbf{Mozaiką} nazywa się obraz, który powstaje poprzez połączenie grupy obrazów zwanych kafelkami na podstawie ich wzajemnych relacji. Znanym oraz popularnym przykładem łączenia obrazów w jeden większy jest funkcja panoramy w telefonach komórkowych, czy aparatach fotograficznych. Od strony użytkownika proces tworzenia panoramy polega na powolnym przesuwaniu telefonem po linii poziomej do momentu aż żądany krajobraz zostanie uchwycony. Od strony urządzenia proces polega na wykonywaniu serii zdjęć oraz następnie łączenie nachodzących klatek w jeden obraz. Rezultatem jest jednolita panorama, która składa się z grupy mniejszych węższych zdjęć.

Celem niniejszej pracy jest stworzenie mozaiki OCT (przykład na rysunku \ref{fig:algorytmy_korejestracji:mosaic}), która powstaje z połączenia mniejszych nachodzących na siebie nawzajem angiograficznych obrazów OCT (przykład obrazu angiograficznego OCT znajduje się z prawej strony na rysunku \ref{fig:obrazowanie_oct:bscan_vessels}).

\begin{figure}[H]
  \centering
  \includegraphics[width=10cm]{gfx/mosaic}
  \caption{Mozaika OCT stworzona z połączenia angiograficznych obrazów OCT.}
  \label{fig:algorytmy_korejestracji:mosaic}
\end{figure}

Proces automatycznego stworzenia mozaiki takiej jak na rysunku \ref{fig:algorytmy_korejestracji:mosaic} jest zadaniem nietrywialnym i wymaga dokładnej analizy wiedzy dziedzinowej oraz precyzyjnego wyboru metod. Pierwszym krokiem jest wybór modelu deformacji kafelków (sekcja \ref{sec:algorytmy_korejestracji:model_deformacji}), następnym etapem, który jest jednocześnie najbardziej wymagającym jest wybór metody wzajemnej korejestracji kafelków (sekcja \ref{sec:algorytmy_korejestracji:korejestracja_kafelow}). Posiadając zdefiniowane wzajemne relacje kafelków oraz ich docelowe położenie w finalnej mozaice należy wykonać proces łączenia kafelków (sekcja \ref{sec:algorytmy_korejestracji:laczenie_kafelkow}). W każdej z tych sekcji została wyjaśniona idea metody w kontekście stworzenia mozaiki OCT, natomiast szczegółowy opis zaimplementowanych metod znajduje się w rozdziale \ref{sec:proponowane_algorytmy}.

\section{Model deformacji kafelków}
\label{sec:algorytmy_korejestracji:model_deformacji}

Model deformacji kafelków określa dozwolone przekształcenia geometryczne, które odwzorują piksele kafelka do pikseli kafelka w finalnej mozaice. Ze względu na to, że angiograficzne obrazy OCT znajdują się na jednej płaszczyźnie możliwy zbiór modeli deformacji ogranicza nam się do transformacji dwuwymiarowych, które zostały zobrazowane na rysunku \ref{fig:algorytmy_korejestracji:trans}.

\begin{figure}[H]
  \centering
  \includegraphics[width=\textwidth]{gfx/trans}
  \caption{Zbiór transformacji dwuwymiarowych dla przykładowego angiograficznego obrazu OCT.}
  \label{fig:algorytmy_korejestracji:trans}
\end{figure}

Idealnie OCT powinno tworzyć angiograficzne obrazy, które są względem siebie tylko przesunięte, natomiast w rzeczywistości pojawiają się zniekształcenia wynikające z niedokładności urządzenia oraz ruchu oka pacjenta, przez co niektóre kafelki są nieznacznie obrócone względem siebie. Z tego względu wybranym modelem deformacji kafelków został \textbf{model transformacji ciała sztywnego}, czyli połączenie translacji i rotacji.

\subsection{Matematyczny zapis modelu transformacji ciała sztywnego}

Współrzędne piksela w kafelku możemy określić jako trójelementowy wektor $\widetilde{x}=(x, y, 1)$, gdzie $x$ i $y$ to współrzędne piksela w układzie współrzędnych kafelka. Tak zdefiniowany piksel poddaje się transformacji by uzyskać współrzędne tego piksela $\hat{x}=(x', y')$ w układzie współrzędnych finalnej mozaiki. W sekcji \ref{sec:algorytmy_korejestracji:model_deformacji} została wybrana transformacja ciała sztywnego, który zakłada tylko translację oraz rotację i może być zapisana jako:

\begin{equation}
\hat{x}=\begin{bmatrix}R&t\end{bmatrix}\widetilde{x}
\label{eq:transformation}
\end{equation}

gdzie:

\begin{align}
R &= \begin{bmatrix}cos(\theta)&-sin(\theta)\\sin(\theta)&cos(\theta)\end{bmatrix} &&\text{i} & t &= \begin{bmatrix}t_{x}\\t_{y}\end{bmatrix}
\label{eq:rotation_and_translation}
\end{align}

W równaniu \ref{eq:rotation_and_translation} $\theta$ to kąt obrotu (rotacja) względem początku układu współrzędnych, a $t_{x}$ i $t_{y}$ to odpowiednio przesunięcia względem osi x i osi y. Parametry $\theta$, $t_{x}$ i $t_{y}$ są niewiadomymi równania \ref{eq:transformation}, których obliczenie jest tematem sekcji \ref{sec:algorytmy_korejestracji:korejestracja_kafelow}.

\section{Korejestracja kafelków}
\label{sec:algorytmy_korejestracji:korejestracja_kafelow}

Po wyborze modelu deformacji kafelków można przejść do wyboru metody, która będzie określać jego parametry (w przypadku niniejszej pracy są to parametry przesunięcia i obrotu). Metoda ta powinna zwrócić takie wartości by kafelek znalazł się w odpowiednim miejscu w finalnej mozaice z możliwie najmniejszym błędem. By rozwiązać ten problem najpierw trzeba poznać położenie kafelków względem siebie oraz ustalić kafelek referencyjny. Rysunek \ref{fig:algorytmy_korejestracji:reference_tile} przedstawia przykładowe rozmieszczenie kafelków. Kafelek referencyjny oznaczony poprzez przerywane linie jest przesuwany do finalnej mozaiki, natomiast żeby odpowiednio umieścić kafelki (2) i (3) trzeba najpierw znać dopasowanie kafelków (2) i (3) do kafelka referencyjnego (1). W przykładzie na rysunku \ref{fig:algorytmy_korejestracji:reference_tile} zostało założone, że kafelki (2) i (3) mogą być dopasowane do kafelka (1). Informacja na temat tego, które kafelki należy dopasować do których kafelek wynika z wiedzy dziedzinowej i opisane jest to w sekcji \ref{sec:proponowane_algorytmy:wiedza_dziedzinowa}.

\begin{figure}[H]
  \centering
  \includegraphics[width=\textwidth]{gfx/reference_tile}
  \caption{Przykładowe rozmieszczenie kafelków na finalnej mozaice. Kafelek referencyjny został wyróżniony przerywaną linią.}
  \label{fig:algorytmy_korejestracji:reference_tile}
\end{figure}

Dopasowanie dwóch kafelków do siebie nazywa się również ich korejestracją, czyli przeniesieniem dwóch kafelek do wspólnego układu współrzędnych w taki sposób by były względem siebie dopasowane. Prosty przykład korejestracji przedstawiony jest na rysunku \ref{fig:algorytmy_korejestracji:align}, gdzie dwie kafelki mają wspólny obszar nałożenia.

\begin{figure}[H]
  \centering
  \includegraphics[width=7cm]{gfx/align}
  \caption{Przykładowa korejestracja dwóch kafelek ze wspólnym obszarem nałożenia.}
  \label{fig:algorytmy_korejestracji:align}
\end{figure}

Dopasowanie kafelków z rysunku \ref{fig:algorytmy_korejestracji:align} jest przykładem bardzo prostym, ponieważ wymaga tylko translacji jednej kafelki względem drugiej. W rzeczywistości pomiary OCT nie są tak dokładne przez co prosta translacja wzdłuż jednej z osi nie wystarcza i wymagane jest automatyczne wyznaczenie translacji wzdłuż z każdej z osi oraz parametru rotacji. Algorytmy korejestracji umożliwiające automatyczne dopasowanie można podzielić na dwie dziedziny:

\begin{enumerate}
\item Korejestracja na podstawie wartości pikseli (bezpośrednia).
\item Korejestracja na podstawie cech.
\end{enumerate}

\subsection{Korejestracja na podstawie wartości pikseli (bezpośrednie)}
\label{sec:algorytmy_korejestracji:korejestracja_na_podstawie_wartosci}

\subsection{Korejestracja na podstawie cech}
\label{sec:algorytmy_korejestracji:korejestracja_na_podstawie_cech}

\section{Łączenie kafelków}
\label{sec:algorytmy_korejestracji:laczenie_kafelkow}























% !TEX root = ../thesis-example.tex
%
\chapter{Proponowane algorytmy}
\label{sec:proponowane_algorytmy}

Niniejszy rozdział opisuje szczegółowo kolejne kroki rozwiązania problemu, który został dokładnie przestawiony w sekcji \ref{sec:wstep:opis_problemu}. Sekcja \ref{sec:proponowane_algorytmy:implementacja} opisuje sposób implementacji metod oraz wykorzystane technologie. Sekcja \ref{sec:proponowane_algorytmy:wiedza_dziedzinowa} opisuje wiedzę dziedzinową na temat danych wejściowych (kafelków). Następnie sekcje od \ref{sec:proponowane_algorytmy:sift} do \ref{sec:proponowane_algorytmy:laczenie_kafelkow} opisują zasadę działania poszczególnych metod w kolejności zgodnej z ich wykonywaniem w programie.

\section{Implementacja}
\label{sec:proponowane_algorytmy:implementacja}

% W jaki sposób zostało to zaimplementowane

\section{Wiedza dziedzinowa}
\label{sec:proponowane_algorytmy:wiedza_dziedzinowa}

% Co dostaje w wiedzy dziedzinowej

\section{Rejestracja kafelków poprzez ekstrakcję cech SIFT}
\label{sec:proponowane_algorytmy:sift}

\subsection{Dopasowanie wyekstrahowanych cech}
\label{sec:proponowane_algorytmy:filtrowanie}

\subsection{Filtrowanie dopasowań}
\label{sec:proponowane_algorytmy:filtrowanie}

\subsubsection{Filtrowanie na podstawie wiedzy dziedzinowej}
\label{sec:proponowane_algorytmy:filtrowanie_dziedzinowe}

\subsubsection{RANSAC}
\label{sec:proponowane_algorytmy:ransac}

\section{Rejestracja kafelków poprzez wykrycie położeń naczyń krwionośnych w kafelkach}
\label{sec:proponowane_algorytmy:depth_first_search}

\section{Estymacja macierzy transformacji pomiędzy kafelkami}
\label{sec:proponowane_algorytmy:estymacja}

\section{Globalna rejestracja kafelków}
\label{sec:proponowane_algorytmy:globalna_rejestracja}

\section{Łączenie kafelków}
\label{sec:proponowane_algorytmy:laczenie_kafelkow}

%\texttt{ code }

% !TEX root = ../thesis.tex
%
\chapter{Oprogramowanie}
\label{sec:oprogramowanie:oprogramowanie}

Program o nazwie \texttt{mostitch}, który jest celem niniejszej pracy jest całkowicie napisany w języku C++. Język wybrano ze względu na to, że wykorzystywana biblioteka przetwarzania obrazów OpenCV (opisana w skrócie w sekcji \ref{sec:oprogramowanie:opencv}) posiada interfejs w języku C++. Program można ściągnąć z repozytorium\footnote{\url{http://git.tesla.cs.put.poznan.pl/agrzyb/mostitch/tree/master}}, a następnie zainstalować postępując zgodnie z instrukcją napisaną w pliku \texttt{README.md}. Po udanej instalacji program można uruchomić za pomocą komendy:

\begin{verbatim}
mostitch path_to_config_file
\end{verbatim}

gdzie \texttt{path\_to\_config\_file} to obowiązkowy argument do programu wskazujący ścieżkę do pliku konfiguracyjnego (opisanego dokładniej w sekcji \ref{sec:oprogramowanie:plik_konfiguracyjny}), który określa wszystkie parametry niezbędne do prawidłowego działania programu.

Program umożliwia konstrukcję mozaik dla dowolnej ilości zbiorów kafelków. Dodatkowo dla każdego zbioru kafelków powstają różne wersje mozaik, ze względu na wykorzystanie różnych metod ich tworzenia. Dzięki tej funkcjonalności użytkownik może wybrać najbardziej odpowiednią mozaikę. Wyjściem programu więc jest zbiór mozaik (obrazy \texttt{.png}), w którym każda mozaika ma przypisaną metodę jej tworzenia oraz odpowiadający zbiór kafelków.

\section{Plik konfiguracyjny}
\label{sec:oprogramowanie:plik_konfiguracyjny}

Do zarządzania plikiem konfiguracyjnym wykorzystano bibliotekę \textbf{libconfig}\footnote{\url{http://www.hyperrealm.com/libconfig/}} umożliwiającą bezproblemowy odczyt pliku konfiguracyjnego z rozszerzeniem \texttt{.cfg}. Format pliku konfiguracyjnego jest bardziej czytelny w porównaniu do powszechnie wykorzystywanych plików XML. Biblioteka również jest świadoma typu zmiennej przez co unika się konwersji typu \texttt{string} na typy takie jak \texttt{int}, czy \texttt{float}.

W pliku konfiguracyjnym zawarte są informacje:

\begin{itemize}
\item Ścieżka do miejsca z folderami zawierającymi zbiory kafelków do złączenia.
\item Ścieżka do miejsca, w którym będą zapisane wynikowe mozaiki.
\item Parametry pozwalające na automatyczne wczytanie obrazów kafelków.
\item Parametry modyfikujące działanie metod przetwarzania kafelków.
\end{itemize}

Wszystkie parametry zawarte w pliku konfiguracyjnym są szczegółowo opisane w przykładzie pliku konfiguracyjnego dołączonego do niniejszej pracy o nazwie \texttt{config.cfg}.

\section{Dokumentacja}
\label{sec:oprogramowanie:dokumentacja}

Dokumentacje programu przygotowano za pomocą narzędzia do generacji dokumentacji \textbf{Doxygen}\footnote{\url{http://www.stack.nl/~dimitri/doxygen/}}. Żeby wyświetlić dokumentację należy otworzyć plik \texttt{index.html} w przeglądarce znajdujący się w folderze \texttt{docs} w repozytorium\footnote{\url{http://git.tesla.cs.put.poznan.pl/agrzyb/mostitch/tree/master}} programu.

\section{OpenCV}
\label{sec:oprogramowanie:opencv}

Wszystkie rozwiązania zaimplementowane w niniejszej pracy bazują na bibliotece przetwarzania obrazów o nazwie \textbf{OpenCV}. Biblioteka jest dozwolona do bezpłatnego wykorzystania w projektach prywatnych i komercyjnych. OpenCV jest używane w ogromnej ilości projektów z różnych dziedzin i bibliotekę ściągnięto już ponad 9 milionów razy\footnote{\url{http://opencv.org}}. OpenCV cieszy się taką popularnością ze względu na szybkość działania, implementację większości metod przetwarzania obrazu oraz umiejętnością pracy na wielu rdzeniach. Kod źródłowy biblioteki jest dostępny publicznie w serwisie Github\footnote{\url{https://github.com/Itseez/opencv}} przez co każdy może rozwijać OpenCV i ma wgląd do zaimplementowanych metod.

% !TEX root = ../thesis.tex
%
\chapter{Wyniki}
\label{sec:wyniki_eksperymentow}

W niniejszym rozdziale przedstawiono wynik działania programu \texttt{mostitch} na trzech przykładowych zbiorach angiograficznych obrazów OCT dostarczonych w ramach projektu RIMO-BIOL (sekcja \ref{sec:wstep:rimo-biol}). Wynik działania programu na zbiorze to cztery mozaiki powstałe za pomocą różnych kombinacji wartości parametrów (proces opisano w \ref{sec:proponowane_algorytmy:proces_decyzyjny}):

\begin{enumerate}
\item \textbf{Wersja 1} -- \texttt{simplerTransform = false}, \texttt{rigidTransform = true}, \texttt{usePaths = false}. 
\item \textbf{Wersja 2} -- \texttt{simplerTransform = true}, \texttt{rigidTransform = true}, \texttt{usePaths = true}.
\item \textbf{Wersja 3} -- \texttt{simplerTransform = true}, \texttt{rigidTransform = true}, \texttt{usePaths = false}.
\item \textbf{Wersja 4} -- \texttt{simplerTransform = true}, \texttt{rigidTransform = false}, \texttt{usePaths = false}.
\end{enumerate}

\section{Zbiór 1}
\label{sec:zbior_1}

Rysunek \ref{fig:wyniki_eksperymentow:zbior_1} przedstawia zbiór angiograficznych obrazów OCT, natomiast rysunek \ref{fig:wyniki_eksperymentow:wynik_zbior_1} przedstawia wynik działania programu \texttt{mostitch}.

\begin{figure}[H]
  \centering
  \includegraphics[width=5cm]{gfx/zbior_1}
  \caption{Przykładowy zbiór angiograficznych obrazów OCT umieszczonych zgodnie z ich współrzędnymi. Obraz referencyjny znajduje się w prawym górnym rogu.}
  \label{fig:wyniki_eksperymentow:zbior_1}
\end{figure}

\begin{figure}[htb]
  \centering
  \includegraphics[width=10cm]{gfx/wynik_zbior_1}
  \caption{Cztery mozaiki będą wynikiem działania programu \texttt{mostitch} na zbiorze obrazów OCT z rysunku \ref{fig:wyniki_eksperymentow:zbior_1}.}
  \label{fig:wyniki_eksperymentow:wynik_zbior_1}
\end{figure}

\section{Zbiór 2}
\label{sec:zbior_2}

Rysunek \ref{fig:wyniki_eksperymentow:zbior_2} przedstawia zbiór angiograficznych obrazów OCT, natomiast rysunek \ref{fig:wyniki_eksperymentow:wynik_zbior_2} przedstawia wynik działania programu \texttt{mostitch}.

\begin{figure}[H]
  \centering
  \includegraphics[width=5cm]{gfx/zbior_2}
  \caption{Przykładowy zbiór angiograficznych obrazów OCT umieszczonych zgodnie z ich współrzędnymi. Obraz referencyjny znajduje się w prawym górnym rogu.}
  \label{fig:wyniki_eksperymentow:zbior_2}
\end{figure}

\begin{figure}[htb]
  \centering
  \includegraphics[width=10cm]{gfx/wynik_zbior_2}
  \caption{Cztery mozaiki będą wynikiem działania programu \texttt{mostitch} na zbiorze obrazów OCT z rysunku \ref{fig:wyniki_eksperymentow:zbior_2}.}
  \label{fig:wyniki_eksperymentow:wynik_zbior_2}
\end{figure}

\section{Zbiór 3}
\label{sec:zbior_3}

Rysunek \ref{fig:wyniki_eksperymentow:zbior_3} przedstawia zbiór angiograficznych obrazów OCT, natomiast rysunek \ref{fig:wyniki_eksperymentow:wynik_zbior_3} przedstawia wynik działania programu \texttt{mostitch}.

\begin{figure}[H]
  \centering
  \includegraphics[width=5cm]{gfx/zbior_3}
  \caption{Przykładowy zbiór angiograficznych obrazów OCT umieszczonych zgodnie z ich współrzędnymi. Obraz referencyjny znajduje się w prawym górnym rogu.}
  \label{fig:wyniki_eksperymentow:zbior_3}
\end{figure}

\begin{figure}[htb]
  \centering
  \includegraphics[width=10cm]{gfx/wynik_zbior_3}
  \caption{Cztery mozaiki będą wynikiem działania programu \texttt{mostitch} na zbiorze obrazów OCT z rysunku \ref{fig:wyniki_eksperymentow:zbior_3}.}
  \label{fig:wyniki_eksperymentow:wynik_zbior_3}
\end{figure}

\section{Ocena wyników}
\label{sec:ocena_wynikow}

W niniejszej sekcji przeprowadzono analizę wyników rozpoczynając od oceny wizualnej `na oko` w sekcji \ref{sec:wyniki_eksperymentow:ocena_wizualna}, następnie mozaiki oceniano poprzez policzenie odchylenia standardowego w miejscach nałożenia kafelków w sekcji \ref{sec:wyniki_eksperymentow:ochylenie_standardowe}. Czas wykonywania programu \texttt{mostitch} przedstawiono w sekcji \ref{sec:wyniki_eksperymentow:czas_wykonywania}. Na koniec podsumowano wyniki w sekcji \ref{sec:wyniki_eksperymentow:podsumowanie}.

\subsection{Ocena wizualna}
\label{sec:wyniki_eksperymentow:ocena_wizualna}

Na podstawie zaprezentowanych wyników dla trzech przykładowych zbiorów trudno wybrać wersję sprawdzającą się najlepiej dla każdego zbioru. Dla zbioru pierwszego z sekcji \ref{sec:zbior_1} najlepszą wersją jest wersja pierwsza, natomiast dla zbioru drugiego z sekcji \ref{sec:zbior_2} najlepiej prezentującym się wynikiem jest wersja druga lub trzecia. Dla zbioru trzeciego z sekcji \ref{sec:zbior_3} najlepiej wygląda mozaika stworzona wersją drugą lub trzecią. 

\subsection{Ocena na podstawie odchylenia standardowego}
\label{sec:wyniki_eksperymentow:ochylenie_standardowe}

Chcąc stworzyć wiarygodną miarę jakości końcowej mozaiki trzeba najpierw zdefiniować wzorzec, czyli mozaikę złożoną idealnie. W przypadku tematu niniejszej pracy definicja takiego wzorca jest problemem nietrywialnym ze względu na unikalne artefakty występujące w obrazach. Nie jest możliwym by nałożyć kafelki na siebie w taki sposób by piksele idealnie się pokrywały pod względem wartości. Z tego względu końcowa jakość mozaiki $Q$ to suma odchyleń standardowych wartości pikseli w $n$ miejscach gdzie kafelki nakładają się na siebie w mozaice:

\begin{equation}
Q = \sum_{i=1}^{n} \sigma_{i}
\label{eq:standard_deviation}
\end{equation}

Gdzie $\sigma_{i}$ to wartość odchylenia standardowego wartości pikseli kafelków nakładających się w miejscu $i$. Wykres na rysunku \ref{fig:wyniki_eksperymentow:odchylenie_standardowe} przedstawia wartość jakości finalnych mozaik przedstawionych w niniejszym rozdziale dla czterech wersji tworzenia mozaiki.

\begin{figure}[htb]
  \centering
  \includegraphics[width=\textwidth]{gfx/odchylenie_standardowe}
  \caption{Jakość finalnych mozaik przedstawionych w sekcjach \ref{sec:zbior_1}, \ref{sec:zbior_2} i \ref{sec:zbior_3} (im mniej tym lepiej).}
  \label{fig:wyniki_eksperymentow:odchylenie_standardowe}
\end{figure}

Wartości jakości odzwierciedlają ocenę wizualną (sekcja \ref{sec:wyniki_eksperymentow:ocena_wizualna}) za wyjątkiem zbioru pierwszego. Warto również zauważyć, że dla każdego zbioru wersja druga i trzecia zwracają najmniejszą (najlepszą) wartość jakości oraz jest to taka sama wartość. Wersja druga i trzecia różni się występowaniem w procesie algorytmu detekcji naczyń krwionośnych (sekcja \ref{sec:proponowane_algorytmy:depth_first_search}). W wersji drugiej, w której występuje, algorytm najwyraźniej nie zwrócił macierzy transformacji na podstawie wykrytych naczyń krwionośnych, przez co wersja druga jest identyczna do wersji trzeciej (proces dokładniej opisano w sekcji \ref{sec:proponowane_algorytmy:proces_decyzyjny}).

\subsection{Ocena na podstawie czasu wykonywania}
\label{sec:wyniki_eksperymentow:czas_wykonywania}

Wykres na rysunku \ref{fig:wyniki_eksperymentow:czas_wykonywania} przedstawia czas wykonywaniu programu \texttt{mostitch} na zbiorach przedstawionych w niniejszym rozdziale dla czterech wersji tworzenia mozaiki.

\begin{figure}[htb]
  \centering
  \includegraphics[width=\textwidth]{gfx/czas_wykonywania}
  \caption{Czas wykonywania programu \texttt{mostitch} na zbiorach przedstawionych w sekcjach \ref{sec:zbior_1}, \ref{sec:zbior_2} i \ref{sec:zbior_3}.}
  \label{fig:wyniki_eksperymentow:czas_wykonywania}
\end{figure}

Program \texttt{mostitch} najszybciej ($0,332853s$) stworzył mozaikę ze zbioru trzeciego za pomocą wersji czwartej. Na podstawie wyników zaprezentowanych na rysunku \ref{fig:wyniki_eksperymentow:czas_wykonywania} ciężko wybrać wersję, która jest najszybsza, albo najwolniejsza. Wynika to z tego, że dla każdej wersji większość skomplikowanych obliczeń pomimo ustawionych parametrów jest wykonywana. Jest to cecha programu, która na pewno może być ulepszona w przyszłości. Czasy wykonywania programu pomiędzy różnymi zbiorami również są bardzo podobne. Ta kwestia nie powinna dziwić, ponieważ każdy kafelek jest tego samego rozmiaru oraz posiada podobną ilość danych. 

\subsection{Podsumowanie}
\label{sec:wyniki_eksperymentow:podsumowanie}

Podsumowując analiza wizualna (sekcja \ref{sec:wyniki_eksperymentow:ocena_wizualna}), oraz analiza na podstawie czasu wykonania (sekcja \ref{sec:wyniki_eksperymentow:czas_wykonywania}) nie wyznaczyły jednoznacznie wersji, która osiąga najlepsze rezultaty. Analiza z wykorzystaniem odchylenia standardowego (sekcja \ref{sec:wyniki_eksperymentow:ochylenie_standardowe}) wyłoniła na faworytów wersję drugą oraz trzecią, natomiast pomyliła się co do zbioru pierwszego wskazując wersję drugą jako najlepszą, gdzie bezkonkurencyjnie wersja pierwsza stworzyła najlepszą mozaikę. Na podstawie trzech przeprowadzonych analiz nie jest możliwym wyłonić najlepszą wersję. Rozbieżność rezultatów dla różnych wartości parametrów jest głównym powodem tworzenia czterech różnych wersji mozaik przez program \texttt{mostitch}. Dzięki takiemu rozwiązaniu osoba z wiedzą ekspercką jest w stanie stwierdzić, który obraz nadaje się najlepiej do wybranego zastosowania. Co więcej, otrzymanie czterech różnych wynikowych mozaik oszczędza czas uruchamiania programu z różnymi wartościami parametrów.














% !TEX root = ../thesis.tex
%
\chapter{Podsumowanie i wnioski końcowe}
\label{sec:podsumowanie_i_wnioski}

Cel niniejszej pracy został w pełni zrealizowany. Mozaiki obrazów angiograficznych stworzone za pomocą programu \texttt{mostitch} są jednolite oraz spełniają kryteria jakościowe. Cele szczegółowe z sekcji \ref{sec:wstep:cele_szczegolowe} zostały również spełnione:

\begin{enumerate}
\item Zapoznano się z istniejącymi metodami i algorytmami realizującymi \textit{stitching}. Finalnie wybrano korejestrację na podstawie cech SIFT z autorskimi modyfikacjami.
\item Wybrano najbardziej popularną bibliotekę przetwarzania obrazów OpenCV. Środowiskiem deweloperskim został Xcode.
\item Oprogramowanie zostało napisane w języku C++. Program został przetestowany na testowych zbiorach obrazów angiograficznych OCT dostarczonych w ramach projektu RIMO-BIOL.
\item Program \texttt{mostitch} został zaimplementowany oraz zbudowany. Działa na komputerach posiadających system operacyjny OS X lub Linux.
\item Dokumentacja oprogramowania jest zawarta w plikach źródłowych programu \texttt{mostitch}, który jest zapisany na płycie CD dołączonej do niniejszej pracy.
\end{enumerate}

\section{Napotkane trudności}
\label{sec:podsumowanie_i_wnioski:napotkane_trudnosci}

Podczas realizacji celu niniejszej pracy zdecydowanie największy okres czasu spędzono na implementacji oraz testowaniu programu \texttt{mostitch}. W większości praca była usystematyzowana i poszczególne zadania dostarczano na wyznaczony termin. W trakcie tworzenia programu \texttt{mostitch} można wyznaczyć dwa przypadki w których praca trwała dłużej niż to pierwotnie zakładano:

\begin{enumerate}
\item Bardzo ważnym czynnikiem wpływającym na rezultat oraz przyszłość pracy był wybór algorytmu ekstrakcji cech w obrazach angiograficznych. Po zapoznaniu się z większością dostępnych algorytmów ostatecznie zdecydowana się na algorytm SIFT, który dostarczył bardzo dobre wyniki.
\item Na początku zakładano, że obrazy angiograficzne będą w rozmiarze 240 na 240 pikseli. Na tym założeniu wybrano algorytm SIFT oraz inne rozwiązania. W trakcie projektu natomiast pojawił się zbiór z obrazami 24 na 240 pikseli. Wykorzystanie wówczas zaimplementowanego algorytmu dało bardzo złe rezultaty, przez co trzeba było stworzyć nowe metody i rozwiązania. Powstał autorski algorytm opisany w sekcji \ref{sec:proponowane_algorytmy:depth_first_search}.
\end{enumerate}

\section{Możliwości rozwoju}
\label{sec:podsumowanie_i_wnioski:mozliwosci_rozwoju}

Program \texttt{mostitch} ma duży potencjał i poprzez modularną budowę jego elementy mogą być wykorzystane w innych projektach w przyszłości. Na obecną chwilę istnieje kilka pomysłów na jego ulepszenia i rozwój:

\begin{itemize}
\item Dołączenie innych algorytmów ekstrahujących cechy w angiograficznych obrazach OCT (np. SURF, FAST), a następnie porównanie wynikowych mozaik.
\item Implementacja programu okienkowego, w którym użytkownik poprzez przyjazny interfejs ładowałby angiograficzne obrazy z dysku, a następnie przeciągając obrazy ustalałby ich relacje przestrzenne. Takie rozwiązanie wyeliminowałoby plik konfiguracyjny, który jest mniej przyjazny dla użytkownika.
\item W trakcie tworzenia pracy zdecydowano się na model transformacji bryły sztywnej (translacja i rotacja) ze względu na to, że operacja skalowania generowała zbyt dużo błędów. Czasem jednak lekkie skalowanie jest niezbędne by uzyskać pożądany wynik. Dodanie operacji skalowania, której parametry byłyby mocno ograniczone i kontrolowane przyczyniłoby się do poprawy jakości mozaik.
\end{itemize}



\cleardoublepage

% --------------------------
% Back matter
% --------------------------
{%
\setstretch{1.1}
\renewcommand{\bibfont}{\normalfont\small}
\setlength{\biblabelsep}{0pt}
\setlength{\bibitemsep}{0.5\baselineskip plus 0.5\baselineskip}
\printbibliography[nottype=online]
\printbibliography[heading=subbibliography,title={Strony internetowe},type=online,prefixnumbers={@}]
}
\cleardoublepage

\newpage
\mbox{}

% **************************************************
% End of Document CONTENT
% **************************************************
\end{document}
