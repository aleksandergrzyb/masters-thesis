% !TEX root = ../thesis-example.tex
%
\chapter{Proponowane algorytmy}
\label{sec:proponowane_algorytmy}

Niniejszy rozdział opisuje szczegółowo kolejne kroki rozwiązania problemu, który został dokładnie przestawiony w sekcji \ref{sec:wstep:opis_problemu}. Sekcja \ref{sec:proponowane_algorytmy:implementacja} opisuje sposób implementacji metod oraz wykorzystane technologie. Sekcja \ref{sec:proponowane_algorytmy:wiedza_dziedzinowa} opisuje wiedzę dziedzinową na temat danych wejściowych (kafelków). Następnie sekcje od \ref{sec:proponowane_algorytmy:sift} do \ref{sec:proponowane_algorytmy:laczenie_kafelkow} opisują zasadę działania poszczególnych metod w kolejności zgodnej z ich wykonywaniem w programie.

\section{Implementacja}
\label{sec:proponowane_algorytmy:implementacja}

% W jaki sposób zostało to zaimplementowane

\section{Wiedza dziedzinowa}
\label{sec:proponowane_algorytmy:wiedza_dziedzinowa}

% Co dostaje w wiedzy dziedzinowej

\section{Rejestracja kafelków poprzez ekstrakcję cech SIFT}
\label{sec:proponowane_algorytmy:sift}

\subsection{Dopasowanie wyekstrahowanych cech}
\label{sec:proponowane_algorytmy:filtrowanie}

\subsection{Filtrowanie dopasowań}
\label{sec:proponowane_algorytmy:filtrowanie}

\subsubsection{Filtrowanie na podstawie wiedzy dziedzinowej}
\label{sec:proponowane_algorytmy:filtrowanie_dziedzinowe}

\subsubsection{RANSAC}
\label{sec:proponowane_algorytmy:ransac}

\section{Rejestracja kafelków poprzez wykrycie położeń naczyń krwionośnych w kafelkach}
\label{sec:proponowane_algorytmy:depth_first_search}

\section{Estymacja macierzy transformacji pomiędzy kafelkami}
\label{sec:proponowane_algorytmy:estymacja}

\section{Globalna rejestracja kafelków}
\label{sec:proponowane_algorytmy:globalna_rejestracja}

\section{Łączenie kafelków}
\label{sec:proponowane_algorytmy:laczenie_kafelkow}

%\texttt{ code }
