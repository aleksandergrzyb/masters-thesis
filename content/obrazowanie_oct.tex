% !TEX root = ../thesis.tex
%
\chapter{Obrazowanie OCT}
\label{sec:obrazowanie_oct}

% Wstęp do OCT

\textbf{Optyczna tomografia koherencyjna} (ang. \textit{optical coherence tomography, OCT}) jest metodą umożliwiającą nieinwazyjne oraz \textit{in vivo} przechwycenie obrazu wnętrza tkanki biologicznej. Zasada działania OCT opiera się na wykorzystaniu fal świetlnych. Dzięki temu rozdzielczość obrazów jest o wiele wyższa niż w ultrasonografii (wykorzystanie fal dźwiękowych), czy rezonansie magnetycznym (wykorzystanie pola magnetycznego). Następnym powodem dużej popularności OCT w medycynie jest bezkontaktowe badanie pacjenta oraz brak wymogu przygotowania pacjenta przed badaniem.

W projekcie RIMO-BIOL (sekcja \ref{sec:wstep:rimo-biol}) OCT wykorzystano do uzyskania szczegółowych obrazów naczyń krwionośnych siatkówki oka. Rysunek \ref{fig:obrazowanie_oct:bscan_vessels} przedstawia przykładowe obrazy siatkówki oka uzyskane poprzez wykorzystanie OCT. Jednym z tych przykładowych obrazów jest angiografia siatkówki oka, która jest obrazem wejściowym do algorytmów omawianych w niniejszej pracy. Sposób powstania angiografii z danych OCT jest wyjaśniony w sekcji \ref{sec:obrazowanie_oct:angiografia_oct}, natomiast ogólna zasada działania OCT jest wyjaśniona w sekcji \ref{sec:obrazowanie_oct:metoda_oct}. Na koniec rozdziału w sekcji \ref{sec:obrazowanie_oct:zastosowania_oct} opisano zastosowania OCT.

\begin{figure}[htb]
	\centering
	\includegraphics[width=\textwidth]{gfx/bscan_vessels}
	\caption{\textbf{Lewy obraz:} Dwuwymiarowy przekrój siatkówki oka (B-skan). Obraz został uzyskany poprzez połączenie jednowymiarowych A-skanów, które zawierają informację o strukturze tkanki w głąb siatkówki oka. \textbf{Prawy obraz:} Angiografia siatkówki oka uzyskana poprzez przetworzenie danych z OCT.}
	\label{fig:obrazowanie_oct:bscan_vessels}
\end{figure}

% Wyjaśnia zasadę działania OCT, bez wchodzenia w duże teoretyczne detale

\section{Zasada działania OCT}
\label{sec:obrazowanie_oct:metoda_oct}

Optyczna tomografia koherencyjna przechwytuje obrazy wnętrza tkanki poprzez wykorzystanie fal świetlnych. OCT za pomocą generatora wytwarza falę świetlną, która jest skierowana na tkankę pacjenta. Następnie po odbiciu fali poprzez tkankę wiązka jest przechwycona przez detektor. Jedną z dostępnych metod, która umożliwiłaby zlokalizowanie miejsca odbicia fali byłoby zmierzenie czasu pomiędzy wygenerowaniem fali, a zarejestrowaniem jej przez detektor (mechanizm stosowany np. w ultrasonografii z wykorzystaniem fal dźwiękowych), natomiast prędkość światła (\(3\times10^8 \frac{m}{s}\)) wyklucza zastosowanie tego mechanizmu. Zjawisko, które umożliwia dokładne zlokalizowanie miejsca odbicia to \textbf{interferencja światła o niskiej spójności}.

\begin{figure}[htb]
	\centering
	\includegraphics[width=\textwidth]{gfx/oct_phases}
	\caption{Kolejne etapy działania metody OCT. \textbf{(1)} - Etap początkowy. \textbf{(2)} - Źródło wyemitowało wiązkę światła. \textbf{(3)} - Fala rozdzieliła się za pomocą interferometru na wiązkę referencyjną (skierowaną na lustro referencyjne) oraz na wiązkę próbki (skierowaną na tkankę). \textbf{(4)} - Wiązki po odbiciu od lustra referencyjnego i tkanki ponownie łączą się za pomocą interferometru. W tej części występuje zjawisko interferencji, które jest zarejestrowane przez detektor.}
	\label{fig:obrazowanie_oct:oct_phases}
\end{figure}

Rysunek \ref{fig:obrazowanie_oct:oct_phases} składa się z bardzo uproszczonych schematów OCT obrazujących kolejne etapy działania metody. Schematy na rysunku \ref{fig:obrazowanie_oct:oct_phases} składają się z pięciu elementów:

\begin{itemize}

\item \textbf{Źródła} - Źródło (np. dioda superluminescencyjna) światła podczerwonego będąca falą o niskiej spójności.
\item \textbf{Rozdzielacza wiązek} (na rysunku \ref{fig:obrazowanie_oct:oct_phases} przedstawiony za pomocą przerywanych linii na środku każdego schematu) - Interferometr (np. Michelsona) umożliwiający rozdzielenie fali na dwie wiązki oraz następne ich połączenie.
\item \textbf{Lustro referencyjne} - Lustro, które odbija wiązkę referencyjną. Posiada możliwość oddalania oraz przybliżania się względem interferometru.
\item \textbf{Tkanka} - Badana tkanka odbijająca wiązkę próbki.
\item \textbf{Detektor} - Rejestruje zjawisko interferencji związek.

\end{itemize}

Najbardziej istotnym etapem wymaganym do zrozumienia mechanizmu OCT jest etap (4) pokazany na rysunku \ref{fig:obrazowanie_oct:oct_phases}. W tym kroku wiązka referencyjna i wiązka próbki łączą się i zachodzi zjawisko interferencji. Dzięki temu, że wiązki są falami o niskiej spójności interferencja zachodzi tylko na małej długości zwanej \textit{coherence length}. Odkrywając za pomocą detektora charakterystyczny wzorzec interferencji występujący na \textit{coherence length} OCT w stanie precyzyjnie odczytać informację odnośnie struktury wnętrza tkanki. Badanie warstw tkanki o różnej głębokości przedstawia rysunek \ref{fig:obrazowanie_oct:tissue_layers}.

\begin{figure}[htb]
	\centering
	\includegraphics[width=\textwidth]{gfx/tissue_layers}
	\caption{Kolejne etapy badania głębszych warstw tkanki dzięki przesuwaniu lustra referencyjnego. Aktualna pozycja lustra referencyjnego jest zaznaczona kolorem czarnym, natomiast aktualnie badana warstwa tkanki jest zaznaczona kolorem czerwonym.}
	\label{fig:obrazowanie_oct:tissue_layers}
\end{figure}

\subsection{Metoda uzyskania trójwymiarowego obrazu tkanki}

Poprzez poruszenie lustra referencyjnego pomiary przeprowadzane są w głąb tkanki (wzdłuż osi Z). Zbiór pomiarów w głąb tkanki nazywa się A-skanem (obraz jednowymiarowy). Powtarzając ten proces w osi X lub Y i następnie poprzez połączenie sąsiadujących A-skanów otrzymuje się przekrój tkanki zwany B-skanem (obraz dwuwymiarowy). Proces uzyskania B-skanów można powtórzyć dla sąsiadujących przekrojów. Poprzez połączenie otrzymanych B-skanów otrzymuje się trójwymiarowy obraz tkanki. Na rysunku \ref{fig:obrazowanie_oct:scan} \cite{Kraus:12} przedstawione są poszczególne skany.

\begin{figure}[htb]
	\centering
	\includegraphics[width=\textwidth]{gfx/scans}
	\caption{\cite{Kraus:12} \textbf{Lewy obraz:} Pojedynczy A-skan wykonany w głąb tkanki. \textbf{Środkowy obraz:} Otrzymany B-skan poprzez połączenie A-skanów. \textbf{Prawy obraz:} Trójwymiarowy obraz tkanki stworzony poprzez połączenie B-skanów.}
	\label{fig:obrazowanie_oct:scan}
\end{figure}

\subsection{OCT w domenie częstotliwości}

Metoda OCT wykonująca ruchy lustrem referencyjnym jest zwana metodą OCT w domenie czasu (ang. \textit{time-domain OCT, TdOCT}). Alternatywną oraz nowszą metodą od TdOCT jest OCT w domenie częstotliwości (ang. \textit{Fourier-domain OCT, FdOCT}). FdOCT umożliwia 100 razy szybsze \cite{Strong:11} skanowanie w porównaniu do TdOCT. TdOCT jest w stanie wykonać ok. 400 A-skanów w przeciągu sekundy, natomiast FdOCT jest ich w stanie wykonać dziesiątki tysięcy. Szybsze skanowanie poprawia również jakość skanów ze względu na to, że pacjent ma mniejszą szansę poruszenia okiem podczas skanowania (ruch oka w trakcie skanowania przyczynia się do powstania artefaktów ruchu na obrazach OCT). Oprócz poprawy szybkości skanowania FdOCT ma wyższą rozdzielczość w przedziale od 3 do 7 mikrometrów \cite{Strong:11}. Jest to poprawa względem TdOCT o 8-10 mikrometrów.

Większa szybkość oraz rozdzielczość FdOCT w porównaniu do TdOCT jest możliwa dzięki dwóm modyfikacjom technicznym:

\begin{enumerate}

\item FdOCT jako źródło światła wykorzystuje laser o wysokiej szerokości pasma, co znacząco zwiększa rozdzielczość.
\item FdOCT jako detektor wykorzystuje spektrometr, który przeprowadza analizę widma fali (połączona wiązka referencyjna i próbki), która dotarła do spektrometru z interferometru. Transformata Fouriera na widmie fali tworzy A-skan tkanki. Dzięki tej technice FdOCT wyeliminowało potrzebę ruszania lustrem referencyjnym co znacząco zwiększa szybkość skanowania.

\end{enumerate}

% Wyjaśnia zasadę powstawania obrazów angiograficznych z danych OCT

\section{Angiografia OCT}
\label{sec:obrazowanie_oct:angiografia_oct}

Angiografia to technika służąca do wizualizacji naczyń krwionośnych i organów ciała. W większości przypadków w medycynie angiografia jest wykonywana poprzez wstrzyknięcie pacjentowi środka kontrastowego, który nie przepuszcza promieni rentgenowskich. Następie by stworzyć obrazy angiograficzne wykorzystuje się jedną z technik obrazowania (np. fluoroskopię) opartą o promienie rentgenowskie. Metoda oparta na stosowaniu środka kontrastowego ma jedną znaczącą wadę, jako technika inwazyjna może prowadzić do reakcji alergicznych na środek kontrastowy oraz jest przeciwwskazana kobietom w ciąży i dzieciom. Z tego powodu nieustannie poszukiwane są metody, które jednocześnie są nieinwazyjne i tworzą obrazy o jakości porównywalnej do metod inwazyjnych.

Angiografia OCT poprzez czasową analizę informacji tomograficznej OCT jest w stanie odtworzyć trójwymiarową informację o strukturze naczyń krwionośnych siatkówki oka. Przykład takiego obrazu znajduje się na rysunku \ref{sec:obrazowanie_oct:angiografia_oct}.

\subsection{Sposób powstania obrazu angiograficznego}

Obrazy angiograficznie użyte w niniejszej pracy powstały za pomocą metody zwanej \textit{Speckle Variance} polegającej na liczeniu wariancji dla każdego piksela z grupy B-skanów zmierzonych w tym samym miejscu w $N$ kolejnych chwilach czasu. Wartość wariancji w obszarach gdzie występuje przepływ krwi (naczynia krwionośne) jest lokalnie wyższa od obszarów, w których znajduje się struktura statyczna. Rezultatem tej metody jest obraz angiograficzny. Na rysunku \ref{fig:obrazowanie_oct:speckle_variance} został przedstawiony przykładowy zbiór B-skanów zmierzonych w tym samym miejscu w kolejnych chwilach czasu.

\begin{figure}[H]
	\centering
	\includegraphics[width=\textwidth]{gfx/speckle_variance}
	\caption{Z lewej strony przedstawiono osiem B-skanów zmierzonych w kolejnych chwilach czasu na postawie których powstaje jeden obraz przepływowy pokazany z prawej strony.}
	\label{fig:obrazowanie_oct:speckle_variance}
\end{figure}

Wartości pikseli $V_{jk}$ obrazu przepływowego liczone są na podstawie wartości pikseli $I_{ijk}$ z grupy $N$ B-skanów za pomocą wzoru (na rysunku \ref{fig:obrazowanie_oct:speckle_variance} $N=8$):
\begin{equation}
V_{jk} = \frac{1}{N} \displaystyle\sum_{i=1}^{N}(I_{ijk} - I_{mean})^2
\end{equation}
Gdzie $j$ i $k$ to współrzędne przestrzenne B-skanu (numery pikseli), a $I_{mean}$ to średnia natężenie grupy pikseli, z których liczona jest wariancja.

\begin{figure}[H]
	\centering
	\includegraphics[width=\textwidth]{gfx/en_face_sv}
	\caption{Z lewej strony znajduje się fragment zbioru sąsiadujących obrazów przepływowych na podstawie których powstaje jeden obraz \textit{en face} pokazany z prawej strony (obraz pochodzi z pracy \cite{Ruminski:15}).}
	\label{fig:obrazowanie_oct:en_face_sv}
\end{figure}

Rezultatem połączenia dwuwymiarowych sąsiadujących obrazów przepływowych jest trójwymiarowy obraz przepływowy, który poddaje się projekcji maksymalnego natężenia (ang. \textit{maximum intensity projection, MIP}) polegającej na projekcji woksela o najwyższej intensywności z obrazu 3D na obraz 2D. Po zastosowaniu projekcji maksymalnego natężenia z trójwymiarowego zestawu danych przepływowych otrzymujemy angiograficzny obraz \textit{en face} naczyń krwionośnych. Rysunek \ref{fig:obrazowanie_oct:en_face_sv} przedstawia część z obrazów przepływowych, na podstawie których stworzono jeden obraz \textit{en face}.

% Opisuje zastosowania OCT

\section{Zastosowania OCT}
\label{sec:obrazowanie_oct:zastosowania_oct}

Optyczna tomografia koherencyjna ze względu na swoje właściwości (badanie nieinwazyjne oraz \textit{in vivo}) jest metodą, która ma szerokie zastosowanie w medycynie oraz w innych specjalizacjach.

\subsection{OCT w okulistyce}

Najbardziej popularnym zastosowaniem OCT w medycynie jest badanie oka \cite{Fercher03}. Technika OCT umożliwia przechwycenie trójwymiarowych obrazów części oka takich jak dno, czy warstwy przednie. Dzięki temu jest wykorzystywane do diagnozowania takich chorób jak stwardnienie rozsiane, zwyrodnienie plamki żółtej, czy jaskra.

\subsection{OCT w gastroenterologii i dermatologii}

OCT w porównaniu do innych metod diagnostycznych w medycynie jest techniką nową. Lekarze i naukowcy nieustannie starają się znaleźć nowe zastosowania dla OCT, która jest metodą obiecującą i szybko rozwijającą się. OCT jest potencjalnym kandydatem by w niektórych diagnozach zastąpić konwencjonalną biopsję wymagającą usunięcia kawałka tkanki z organizmu. Przykładowe zastosowania OCT:

\begin{itemize}
\item \textbf{Badanie struktury błon śluzowych i podśluzowych w układzie pokarmowym.} OCT dostarczyło czyste obrazy dostarczające lekarzom dużo diagnostycznej informacji \cite{Rollins:99}.
\item \textbf{Diagnoza raka skóry} (obecnie diagnozowany jest poprzez biopsję). Niestety OCT w obecnym stopniu zaawansowania nie jest w stanie dostarczyć na tyle dokładnych danych by stać się jedyną metodą diagnozy.
\item \textbf{Diagnoza zapalnych chorób skóry} \cite{Welzel01}.
\end{itemize}

\subsection{OCT w przemyśle}

OCT wykorzystywane jest również w przemyśle. Umożliwia badanie np. grubości materiałów \cite{walecki2006determining}, czy badanie grubości warstwy pancerza tabletek podczas ich produkcji w przemyśle farmaceutycznym \cite{markl2014device}.
